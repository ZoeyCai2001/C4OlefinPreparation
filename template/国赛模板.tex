\documentclass[a4paper,10.5pt]{ctexart}
\usepackage[left=2.50cm, right=2.50cm, top=2.50cm, bottom=2.50cm]{geometry} %页边距
\usepackage{amsmath}
\usepackage{amssymb}
\usepackage{ctex}
\usepackage{braket}
\usepackage[table,xcdraw]{xcolor}
\usepackage[european]{circuitikz}
\usepackage{multirow}
\usepackage{float}
\usepackage{bm}         % 加粗方程字体
\usepackage{graphicx}
\usepackage{geometry}
\usepackage{listings}
\usepackage{xcolor}
\usepackage{helvet}
\usepackage[english]{babel}
\usepackage{booktabs}
\usepackage{indentfirst}
\geometry{left=2.0cm,right=2.0cm,top=2.0cm,bottom=2.0cm}
\usepackage{textcomp}
\usepackage{physics}
%\usepackage{algorithm}
%\usepackage{algorithmic}
\usepackage{longtable}
\usepackage{array}
\usepackage{listings} 
\usepackage{xcolor}
\usepackage{algpseudocode}
\usepackage{epsfig}
\usepackage[boxed]{algorithm2e}
\lstset{
  language=Matlab,  %代码语言使用的是matlab
  frame=shadowbox, %把代码用带有阴影的框圈起来
  rulesepcolor=\color{red!20!green!20!blue!20},%代码块边框为淡青色
  keywordstyle=\color{blue!90}\bfseries, %代码关键字的颜色为蓝色,粗体
  commentstyle=\color{red!10!green!70}\textit,    % 设置代码注释的颜色
  showstringspaces=false,%不显示代码字符串中间的空格标记
  numbers=left, % 显示行号
  numberstyle=\tiny,    % 行号字体
  stringstyle=\ttfamily, % 代码字符串的特殊格式
  breaklines=true, %对过长的代码自动换行
  extendedchars=false,  %解决代码跨页时,章节标题,页眉等汉字不显示的问题
%   escapebegin=\begin{CJK*},escapeend=\end{CJK*},      % 代码中出现中文必须加上,否则报错
  texcl=true}

\newcommand{\PreserveBackslash}[1]{\let\temp=\\#1\let\\=\temp}
\newcolumntype{C}[1]{>{\PreserveBackslash\centering}p{#1}}
\newcolumntype{R}[1]{>{\PreserveBackslash\raggedleft}p{#1}}
\newcolumntype{L}[1]{>{\PreserveBackslash\raggedright}p{#1}}
\renewcommand{\algorithmicrequire}{ \textbf{Input:}}       
\renewcommand{\algorithmicensure}{ \textbf{Initialize:}} 
\renewcommand{\algorithmicreturn}{ \textbf{Output:}}     

%算法格式
\usepackage{fancyhdr}
\usepackage{subfigure}
\pagestyle{plain}

\lhead{}
\chead{}
\lfoot{}
\cfoot{}
\rfoot{}
\usepackage{hyperref} %bookmarks
\hypersetup{colorlinks, bookmarks, unicode} %unicode
\usepackage{multicol}
\lstset{
	backgroundcolor=\color{green!10!blue!15},%代码块背景色
	rulesepcolor= \color{red!40!blue!100}, %代码块边框颜色
	breaklines=true,  %代码过长则换行
	numbers=left, %行号在左侧显示
	numberstyle= \small,%行号字体
	keywordstyle= \color{blue},%关键字颜色
	commentstyle=\color{gray}, %注释颜色
	frame=shadowbox%用方框框住代码块
}




\title{\textbf{基于所使用的主要模型或者方法的xxxx}}
\date{}



\begin{document}
	\maketitle
\begin{center}
\Large{\bf {摘\qquad 要}}
\end{center}


%解决了什么问题、应用了什么方法、得到了什么结果;文字简练,突出新见解、新方法;最后书写(1000)
\par 开头段:充分概括论文内容,两到三句话:简单交代题目背景;交代我们所做的事情(重要);这个问题的实际意义(可省)
\par \textbf{对于问题一:}解决了什么问题、应用了什么方法(一定要结合题目说明)、得到了什么结果(计算数值答案/开放题简要答主要部分)
\par \textbf{对于问题二:}
\par \textbf{对于问题三:}
\par 结尾段:
\\


		\noindent{\bf 关键词:}关键词1\quad 关键词2\quad 关键词3\quad 关键词4
\clearpage



\section{问题重述}
%数学建模比赛论文是要我们解决一道给定的问题,所以正文部分一般应从问题重述开始,一般确定选题后就可以开始写这一部分了。
%这部分的内容是将原问题进行整理,将问题背景和题目分开陈述即可,所以基本没啥难度。
%本部分的目的是要吸引读者读下去,所以文字不可冗长,内容选择不要过于分散、琐碎,措辞要精练。
%注意:在写这部分的内容时,绝对不可照抄原题!(论文会查重)
%应为:在仔细理解了问题的基础上,用自己的语言重新将问题描述一遍。语言需要简明扼要,没有必要像原题一样面面俱到。
\subsection{问题背景}
\par 丰富题目背景,更改原题中的说法
\subsection{问题提出}
\par 结合自己的分析思路来重新描述问题,更改原题中的说法
\section{问题分析}
\subsection{问题一的分析}
\par 这部分的内容应包括:题目中包含的信息和条件,利用信息和条件对题目做整体分析,确定用什么方法建立模型,一般是每个问题单独分析一小节,分析过程要简明扼要, 不需要放结论。

建议在文字说明的同时{\bf 用图形或图表(例如流程图)列出思维过程},这会使你的思维显得很清晰,让人觉得一目了然。
\subsection{问题二的分析}

\subsection{问题三的分析}

\section{模型假设}

\par 1.题目明确给出的假设条件

2.排除生活中的小概率事件(例如黑天鹅事件、非正常情况)

3.仅考虑问题中的核心因素,不考虑次要因素的影响

4.使用的模型中要求的假设

5.对模型中的参数形式(或者分布)进行假设

6.和题目联系很紧密的一些假设,主要是为了简化模型
\section{符号说明}
%本部分是对模型中使用的重要变量进行说明,一般排版时要放到一张表格中。
%注意:第一:不需要把所有变量都放到这个表里面,模型中用到的临时变量可以不放。第二:下文中首次出现这些变量时也要进行解释,不然会降低文章的可读性。

\begin{longtable}[h]
\centering
\begin{tabular}{C{2cm}C{12cm}C{2cm}}
\toprule[2pt]
\textbf {符号}&\textbf {定义}&\textbf {单位}\\\midrule[1pt]
	0         & 0            &   0\\
\bottomrule[2pt]
\end{tabular}
\label{tab:wei1}
\end{longtable}

\section{模型的建立与求解}
\subsection{问题模型一的建立与求解}
\subsubsection{模型的建立}
模型建立是将原问题抽象成用数学语言的表达式,它一定是在先前的问题分析和模型假设的基础上得来的。因为比赛时间很紧,大多时候我们都是使用别人已经建立好的模型。这部分一定要将题目问的问题和模型紧密结合起来,切忌随意套用模型。我们还可以对已有模型的某一方面进行改进或者优化,或者建立不同的模型解决同一个问题,这样就是论文的创新和亮点。
\subsubsection{模型的求解}
把实际问题归结为一定的数学模型后,就要利用数学模型求解所提出的实际问题了。一般需要借助计算机软件进行求解,例如常用的软件有Matlab, Spss, Lingo, Excel, Stata, Python等。求解完成后,得到的求解结果应该规范准确并且醒目,若求解结果过长,最好编入附录里。(注意:如果使用智能优化算法或者数值计算方法求解的话,需要简要阐明算法的计算步骤)
\subsection{问题二模型的建立与求解}
\subsubsection{模型的建立}
\subsubsection{模型的求解}\subsection{问题三模型的建立与求解}
\subsubsection{模型的建立}
\subsubsection{模型的求解}

\section{模型的分析与检验}
模型的分析与检验的内容也可以放到模型的建立与求解部分,这里我们单独抽出来进行讲解,因为这部分往往是论文的加分项,很多优秀论文也会单独抽出一节来对这个内容进行讨论。\cite{article1}

模型的分析 :在建模比赛中模型分析主要有两种,一个是灵敏度(性)分析,另一个是误差分析。灵敏度分析是研究与分析一个系统(或模型)的状态或输出变化对系统参数或周围条件变化的敏感程度的方法。其通用的步骤是:控制其他参数不变的情况下,改变模型中某个重要参数的值,然后观察模型的结果的变化情况。误差分析是指分析模型中的误差来源,或者估算模型中存在的误差,一般用于预测问题或者数值计算类问题。

模型的检验:模型检验可以分为两种,一种是使用模型之前应该进行的检验,例如层次分析法中一致性检验,灰色预测中的准指数规律的检验,这部分内容应该放在模型的建立部分;另一种是使用了模型后对模型的结果进行检验,数模中最常见的是稳定性检验,实际上这里的稳定性检验和前面的灵敏度分析非常类似,等会大家看到例子就明白了。
(大家尽量在论文中使用灵敏度分析,视频中有详细的讲解)

\section{模型的评价、改进与推广}
注:本部分的标题需要根据你的内容进行调整,例如:如果你没有写模型推广的话,就直接把标题写成模型的评价与改进。很多论文也把这部分的内容直接统称为“模型评价”部分,也是可以的。
\subsection{模型的优点}
优缺点是必须要写的内容,改进和推广是可选的,但还是建议大家写,实力比较强的建模者可以在这一块充分发挥,这部分对于整个论文的作用在于画龙点睛。
\subsection{模型的缺点}
缺点写的个数要比优点少
\subsection{模型的改进}
主要是针对模型中缺点有哪些可以改进的地方;
\subsection{模型的推广}
将原题的要求进行扩展,进一步讨论模型的实用性和可行性。
\renewcommand\refname{参考文献}
	\begin{thebibliography}{100}%此处数字为最多可添加的参考文献数量
		\bibitem{article1}文章引用:作者,论文名,杂志名,卷期号:起止页码,出版年。%title author journal data pages
		\bibitem{book1}书籍引用:作者,书名,出版地:出版社,出版年月%title author publish date
	\end{thebibliography}
\clearpage


\begin{center}
\Large{\bf {附\qquad 录}}
\end{center}
\fbox{

  \parbox{1\textwidth}{

  \bf{附录1}\\
   支撑材料的文件列表:
  }
}
\\
\lstset{language=Matlab}
\begin{lstlisting}
%附录2
%代码1:

\end{lstlisting}

\begin{lstlisting}
%附录3
%代码2:

\end{lstlisting}

\end{document}