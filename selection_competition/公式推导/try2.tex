\documentclass[a4paper,10.5pt]{ctexart}
\usepackage[left=2.50cm, right=2.50cm, top=2.50cm, bottom=2.50cm]{geometry} %页边距
\usepackage{amsmath}
\usepackage{amssymb}
\usepackage{ctex}
\usepackage{braket}
\usepackage[table,xcdraw]{xcolor}
\usepackage[european]{circuitikz}
\usepackage{multirow}
\usepackage{float}
\usepackage{bm}         % 加粗方程字体
\usepackage{graphicx}
\usepackage{geometry}
\usepackage{listings}
\usepackage{xcolor}
\usepackage{helvet}
\usepackage[english]{babel}
\usepackage{booktabs}
\usepackage{indentfirst}
\geometry{left=2.0cm,right=2.0cm,top=2.0cm,bottom=2.0cm}
\usepackage{textcomp}
\usepackage{physics}
%\usepackage{algorithm}
%\usepackage{algorithmic}
\usepackage{longtable}
\usepackage{array}
\usepackage{listings} 
\usepackage{xcolor}
\usepackage{algpseudocode}
\usepackage{epsfig}
\usepackage[boxed]{algorithm2e}
\lstset{
  language=Matlab,  %代码语言使用的是matlab
  frame=shadowbox, %把代码用带有阴影的框圈起来
  rulesepcolor=\color{red!20!green!20!blue!20},%代码块边框为淡青色
  keywordstyle=\color{blue!90}\bfseries, %代码关键字的颜色为蓝色,粗体
  commentstyle=\color{red!10!green!70}\textit,    % 设置代码注释的颜色
  showstringspaces=false,%不显示代码字符串中间的空格标记
  numbers=left, % 显示行号
  numberstyle=\tiny,    % 行号字体
  stringstyle=\ttfamily, % 代码字符串的特殊格式
  breaklines=true, %对过长的代码自动换行
  extendedchars=false,  %解决代码跨页时,章节标题,页眉等汉字不显示的问题
%   escapebegin=\begin{CJK*},escapeend=\end{CJK*},      % 代码中出现中文必须加上,否则报错
  texcl=true}

\newcommand{\PreserveBackslash}[1]{\let\temp=\\#1\let\\=\temp}
\newcolumntype{C}[1]{>{\PreserveBackslash\centering}p{#1}}
\newcolumntype{R}[1]{>{\PreserveBackslash\raggedleft}p{#1}}
\newcolumntype{L}[1]{>{\PreserveBackslash\raggedright}p{#1}}
\renewcommand{\algorithmicrequire}{ \textbf{Input:}}       
\renewcommand{\algorithmicensure}{ \textbf{Initialize:}} 
\renewcommand{\algorithmicreturn}{ \textbf{Output:}}     

%算法格式
\usepackage{fancyhdr}
\usepackage{subfigure}
\pagestyle{plain}

\lhead{}
\chead{}
\lfoot{}
\cfoot{}
\rfoot{}
\usepackage{hyperref} %bookmarks
\hypersetup{colorlinks, bookmarks, unicode} %unicode
\usepackage{multicol}
\lstset{
	backgroundcolor=\color{green!10!blue!15},%代码块背景色
	rulesepcolor= \color{red!40!blue!100}, %代码块边框颜色
	breaklines=true,  %代码过长则换行
	numbers=left, %行号在左侧显示
	numberstyle= \small,%行号字体
	keywordstyle= \color{blue},%关键字颜色
	commentstyle=\color{gray}, %注释颜色
	frame=shadowbox%用方框框住代码块
}
\begin{document}
记大圆的圆心坐标为$(x_0,y_0)$,第一台摄像机$m_1$与圆心连线与$x$轴夹角为$\theta$,则$m_1$坐标为$(x_0+50cos\ \theta,y_0+50sin\ \theta)$,设为$(x_1,y_1)$,则过$m_1$所作的椭圆的两条切线与椭圆的切点的连线方程为:$$\frac{x_1x}{l^2}+\frac{y_1y}{k^2}=1$$
\par 联立方程组
\begin{equation}
\nonumber
\left\{
\begin{aligned}
\frac{x_1x}{l^2}+\frac{y_1y}{k^2}=1 \\
\frac{x^2}{l^2}+\frac{y^2}{k^2}=1
\end{aligned}
\right.
\end{equation}
\par 利用$MATLAB$求解得到由$m_1$作切线得到的两个切点$T_1(x_2,y_2)$、$T_2(x_3,y_3)$.
由弦长公式,得到割线$T_1T_2$的长度$L$:
\begin{equation}
\begin{aligned}
L^2=(1+\frac{k^4x_{1}^2}{l^4y_{1}^2})(x_2-x_3)^2
\end{aligned}
\end{equation}
\par 进一步,利用直角三角形$T_1T_2T_3$的各边关系,得到$T_1T_3$即摄像机拍到的线段$a$的表达式为:
\begin{equation}
\begin{aligned}
a&=L·|cos(\phi-\theta-\frac{\pi}{2})|\\
&=L·|sin(\phi-\theta)|
\end{aligned}
\end{equation}
\par 其中,$\phi=arctan(-\frac{k^2x_1}{l^2y_1})$,为割线$T_1T_2$与$x$轴正方向的夹角。

\par 下面考虑照片上所反映的线段的端点偏离圆心O的程度(由端点到圆心的较短距离衡量)。由于$O(x_0,y_0)$,作$Om_1$的延长线与$T_1T_3$交于点Q,则$QT_1$为照片上所反映的线段的端点$T_1$到圆心$0$的距离。下面考虑$QT_1$的长度:
\par $OQ$的直线方程为:$$y-y_0=tan\ \theta(x-x_0)$$
\par $T_1Q$的直线方程为:$$cos\ \theta(x-x_2)+sin\ \theta(y-y_2)=0$$
\par 故而两条直线的交点$Q(x_4,y_4)$坐标为
\begin{equation}
\begin{aligned}
x_4&=\frac{cos\theta x_2-sin\ \theta y_0+sin\theta tan\theta x_0+sin\theta y_2}{cos\theta+sin\theta tan\theta}\\
y_4&=\frac{sin\theta x_2-sin\theta x_0+cos\theta y_0+sin\theta tan\theta y_2}{cos\theta+sin\theta tan\theta}
\end{aligned}
\end{equation}
\par 故而
$$T_1Q=\sqrt{x_4-x_2)^2+(y_4-y_2)^2}$$
\par 同理可以得到另一个端点$T_2$偏离圆心的距离$T_2R=\sqrt{(x_5-x_3)^2+(y_5-y_3)^2}$,其中$R$的垂点$(x_5,y_5)$为:
\begin{equation}
\begin{aligned}
x_4&=\frac{cos\theta x_3-sin\ \theta y_0+sin\theta tan\theta x_0+sin\theta y_3}{cos\theta+sin\theta tan\theta}\\
y_4&=\frac{sin\theta x_3-sin\theta x_0+cos\theta y_0+sin\theta tan\theta y_3}{cos\theta+sin\theta tan\theta}
\end{aligned}
\end{equation}
\par 取二者中较小值作为观测到的线段的端点偏离圆心$O$的程度,即
$$d_1=min\{T_1Q,T_2R\}$$
\par 同时考虑$m_2$与$m_3$观测到的线段长度以及偏离程度,将上式中$\theta$分别替换为$\theta+\frac{\pi}{3}$以及$\theta+\frac{2\pi}{3}$可以得到$b$、$c$以及$d_2$、$d_3$的表达式。
\end{document}