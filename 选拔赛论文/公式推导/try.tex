\documentclass[a4paper,10.5pt]{ctexart}
\usepackage[left=2.50cm, right=2.50cm, top=2.50cm, bottom=2.50cm]{geometry} %页边距
\usepackage{amsmath}
\usepackage{amssymb}
\usepackage{ctex}
\usepackage{braket}
\usepackage[table,xcdraw]{xcolor}
\usepackage[european]{circuitikz}
\usepackage{multirow}
\usepackage{float}
\usepackage{bm}         % 加粗方程字体
\usepackage{graphicx}
\usepackage{geometry}
\usepackage{listings}
\usepackage{xcolor}
\usepackage{helvet}
\usepackage[english]{babel}
\usepackage{booktabs}
\usepackage{indentfirst}
\geometry{left=2.0cm,right=2.0cm,top=2.0cm,bottom=2.0cm}
\usepackage{textcomp}
\usepackage{physics}
%\usepackage{algorithm}
%\usepackage{algorithmic}
\usepackage{longtable}
\usepackage{array}
\usepackage{listings} 
\usepackage{xcolor}
\usepackage{algpseudocode}
\usepackage{epsfig}


\usepackage{algorithm,algpseudocode,float}
\usepackage{lipsum}

\makeatletter
\newenvironment{breakablealgorithm}
  {% \begin{breakablealgorithm}
   \begin{center}
     \refstepcounter{algorithm}% New algorithm
     \hrule height.8pt depth0pt \kern2pt% \@fs@pre for \@fs@ruled
     \renewcommand{\caption}[2][\relax]{% Make a new \caption
       {\raggedright\textbf{\ALG@name~\thealgorithm} ##2\par}%
       \ifx\relax##1\relax % #1 is \relax
         \addcontentsline{loa}{algorithm}{\protect\numberline{\thealgorithm}##2}%
       \else % #1 is not \relax
         \addcontentsline{loa}{algorithm}{\protect\numberline{\thealgorithm}##1}%
       \fi
       \kern2pt\hrule\kern2pt
       \renewcommand{\algorithmicensure}{\textbf{Output:}}
     }
  }{% \end{breakablealgorithm}
     \kern2pt\hrule\relax% \@fs@post for \@fs@ruled
   \end{center}
  }
\makeatother
%\usepackage[boxed]{algorithm2e}
\lstset{
  language=Matlab,  %代码语言使用的是matlab
  frame=shadowbox, %把代码用带有阴影的框圈起来
  rulesepcolor=\color{red!20!green!20!blue!20},%代码块边框为淡青色
  keywordstyle=\color{blue!90}\bfseries, %代码关键字的颜色为蓝色,粗体
  commentstyle=\color{red!10!green!70}\textit,    % 设置代码注释的颜色
  showstringspaces=false,%不显示代码字符串中间的空格标记
  numbers=left, % 显示行号
  numberstyle=\tiny,    % 行号字体
  stringstyle=\ttfamily, % 代码字符串的特殊格式
  breaklines=true, %对过长的代码自动换行
  extendedchars=false,  %解决代码跨页时,章节标题,页眉等汉字不显示的问题
%   escapebegin=\begin{CJK*},escapeend=\end{CJK*},      % 代码中出现中文必须加上,否则报错
  texcl=true}

\newcommand{\PreserveBackslash}[1]{\let\temp=\\#1\let\\=\temp}
\newcolumntype{C}[1]{>{\PreserveBackslash\centering}p{#1}}
\newcolumntype{R}[1]{>{\PreserveBackslash\raggedleft}p{#1}}
\newcolumntype{L}[1]{>{\PreserveBackslash\raggedright}p{#1}}
\renewcommand{\algorithmicrequire}{ \textbf{Input:}}       
\renewcommand{\algorithmicensure}{ \textbf{Initialize:}} 
\renewcommand{\algorithmicreturn}{ \textbf{Output:}}     

%算法格式
\usepackage{fancyhdr}
\usepackage{subfigure}
\pagestyle{plain}

\lhead{}
\chead{}
\lfoot{}
\cfoot{}
\rfoot{}
\usepackage{hyperref} %bookmarks
\hypersetup{colorlinks, bookmarks, unicode} %unicode
\usepackage{multicol}
\lstset{
	backgroundcolor=\color{green!10!blue!15},%代码块背景色
	rulesepcolor= \color{red!40!blue!100}, %代码块边框颜色
	breaklines=true,  %代码过长则换行
	numbers=left, %行号在左侧显示
	numberstyle= \small,%行号字体
	keywordstyle= \color{blue},%关键字颜色
	commentstyle=\color{gray}, %注释颜色
	frame=shadowbox%用方框框住代码块
}
\begin{document}
由于三台摄像机被等距地放置在直径为$1m$的圆周上,故而确定一台摄像机在圆周上的位置,即可确定剩余两台摄像机的位置。假设摄像机$m_1$与圆心的连线$l_1$与$x$轴正半轴的夹角分别为$\theta(0\le \theta \le \frac{2\pi}{3} )$,则另外两台摄像机与$x$轴正半轴的夹角为$\theta +\frac{\pi}{3}$与$\theta +\frac{2\pi}{3}$.
\par 首先考虑$\theta \ne \frac{\pi}{2}$且$\theta+\frac{\pi}{3} \ne \frac{\pi}{2}$且$\theta +\frac{2\pi}{3}\ne \frac{3\pi}{2}$则$$l_1:y=tan\theta\cdot x$$
\par 假设所求椭圆长半轴与短半轴长分别为$l$、$k$,则对于椭圆上任意一点$(x_0,y_0)(y_0\ne0)$,过$(x_0,y_0)$且与椭圆相切的直线$l_0$的方程为$$l_0:\frac{x_0x}{l^2}+\frac{y_0y}{k^2}=1$$则$l_0$的斜率为$-\frac{x_0k^2}{y_0l^2}$.
\par 设与$l_1$平行且与椭圆相交的直线分别为$l_2$、$l_3$,它们与椭圆的切点分别为$(x_2,y_2)$、$(x_3,y_3)$,则由$l_1$与$l_2$、$l_3$的斜率相等可得:
\begin{equation}
tan\theta=-\frac{x_2k^2}{y_2l^2}=-\frac{x_3k^2}{y_3l^2}
\end{equation}
\par 由于$(x_2,y_2)$、$(x_3,y_3)$为椭圆上两点,满足方程
\begin{equation}
\begin{align*}
\frac{x_2^2}{l^2}+\frac{y_2^2}{k^2}&=1 \\
 \frac{x_3^2}{l^2}+\frac{y_3^2}{k^2}&=1
\end{align*}
\end{equation}
\par 由$(1)(2)$两式可得:
\begin{equation}
\left\{
\begin{aligned}
x_2^2=x_3^2=\frac{l^4tan^2\theta}{l^2tan^2\theta+k^2} \\
y_2^2=y_3^2=\frac{k^4}{l^2tan^2\theta+k^2}
\end{aligned}
\right.
\end{equation}
\par 由$l_2$的直线方程为$$l_2:\frac{x_2x}{l^2}+\frac{y_2y}{k^2}=1$$
得$l_2$与$l_3$直线之间的距离(即为$(x_3,y_3)$到$l_2$的距离)$d_1$满足:
\begin{align*}
d_1^2&= \frac{|\frac{x_2x_3}{l^2}+\frac{y_2y_3}{k^2}-1|^2}{\frac{x_2^2}{l^4}+\frac{y_2^2}{k^4}}\\
 &=\frac{|-\frac{x_2^2}{l^2}-\frac{y_2^2}{k^2}-1|^2}{\frac{x_2^2}{l^4}+\frac{y_2^2}{k^4}}\\
 &=\frac{4}{\frac{x_2^2}{l^4}+\frac{y_2^2}{k^4}}
\end{align*}
\par 将(3)代入可得:
$$d_1^2=\frac{4(l^2tan^2\theta+k^2)}{1+tan^2\theta}$$
\par 由于$d$即为摄像机所得的线段长度,故而得到三个方程:
\begin{equation}
\left\{
\begin{aligned}
a&=2\sqrt{\frac{l^2tan^2\theta+k^2}{1+tan^2\theta}}\\
b&=2\sqrt{\frac{l^2tan^2(\theta+\frac{\pi}{3})+k^2}{1+tan^2(\theta+\frac{\pi}{3})}}\\
c&=2\sqrt{\frac{l^2tan^2(\theta+\frac{2\pi}{3})+k^2}{1+tan^2(\theta+\frac{2\pi}{3})}}

\end{aligned}
\right.
\end{equation}
\par 再考虑$\theta=\frac{\pi}{2}$,显然有:
\begin{equation}
\left\{
\begin{aligned}
a&=2l\\
b&=\sqrt{l^2+3k^2}\\
c&=\sqrt{l^2+3k^2}
\end{aligned}
\right.
\end{equation}
\par 再考虑$\theta=\frac{\pi}{6}$,显然有:

\begin{equation}
\left\{
\begin{aligned}
a&=\sqrt{l^2+3k^2}\\
b&=2l\\
c&=\sqrt{l^2+3k^2}
\end{aligned}
\right.
\end{equation}

\par 最后考虑$\theta=\frac{5\pi}{6}$,显然有:

\begin{equation}
\left\{
\begin{aligned}
a&=\sqrt{l^2+3k^2}\\
b&=\sqrt{l^2+3k^2}\\
c&=2l
\end{aligned}
\right.
\end{equation}
\par 综上所述:当$a,b,c$互不相等时,可利用$(4)$进行求解,而$a,b,c$中存在两个值相等时,可视相等情况分别利用$(5)(6)(7)$求解。
\par 下利用$Matlab$中$vpasolve$函数对方程进行求解:$vpasolve$函数可以求得代数方程式的数值解。对于含有$m$个未知数$[x_1,x_2,...,x_m]$与$n$个等式方程组$[eq_1,eq_2,...,eq_n]$,函数$vpasolve([eq_1,...,eq_n],[x_1,...,x_m],X_0)$可以返回方程的数值解,其中$X_0$为未知数的初值或者数值解所在的区间。
\par 在$Matlab$中执行以下命令:

\begin{breakablealgorithm}
        \caption{通过$a,b,c$求解椭圆面积$S$的算法}
        \begin{algorithmic}[1] %每行显示行号
            \Require $a,b,c$
            \Ensure  $S$
                    \If {$(a=b\wedge b<\frac{c}{2})\vee (a=c\wedge a<\frac{b}{2})\vee(b=c\wedge b<\frac{a}{2})$}
                    \State return $ -1;$
                    \EndIf
                    \If {$a=b\wedge b=c$}
                    \State $S=\frac{a^2\pi}{4};$
                    \State return $S$
                    \EndIf
                    \If {$a=b\wedge b\ne c$}
                    \State $l=\frac{c}{2};$
                    \State $k=\sqrt{\frac{b^2-l^2}{3}};$
                    \State $S=lk\pi;$
                    \State return $S$;
                    \EndIf
                    \If {$a=c\wedge b\ne c$}
                    \State $l=\frac{b}{2};$
                    \State $k=\sqrt{\frac{a^2-l^2}{3}};$
                    \State $S=lk\pi;$
                    \State return $S$;
                    \EndIf
                    \If {$b=c\wedge a\ne c$}
                    \State $l=\frac{a}{2};$
                    \State $k=\sqrt{\frac{b^2-l^2}{3}};$
                    \State $S=lk\pi;$
                    \State return $S$;
                    \EndIf 
                    \If{$a\ne b\wedge b\ne c\wedge c\ne a$}
                    \State $solve\ equations$
                    \State $a=2\sqrt{\frac{l^2tan^2\theta+k^2}{1+tan^2\theta}}$
                    \State $b=2\sqrt{\frac{l^2tan^2(\theta+\frac{\pi}{3})+k^2}{1+tan^2(\theta+\frac{\pi}{3})}}$
                    \State $c=2\sqrt{\frac{l^2tan^2(\theta+\frac{2\pi}{3})+k^2}{1+tan^2(\theta+\frac{2\pi}{3})}}$           
                    \If {$l\ is\ not\ a\ real\ number$ or $k\ is\  not\  a\  real\  number$}
                    \State return$\ -1;$
                    \Else 
                    \State $S=lk\pi;$
                    \State return $S;$
                    \EndElse
                    \EndIf
                    \EndIf

     \end{algorithmic}
    \end{breakablealgorithm}

\par 通过该程序即可由输入$a$、$b$、$c$得到面积$S$的数值解,无解时返回$-1$。
\par 下面考虑在$Matlab$中使用$solve$函数求得$l,k,\theta$的显式解:$solve$函数与$vpasolve$函数的用法与功能类似,$solve$函数还能模拟人工运算求得函数的公式解。
\par 在$Matlab$中尝试使用$solve$函数求解方程(4),未得到显式解。
\par 再考虑所拍摄的物体为直线或圆的情况:

\begin{itemize}
  \item [1)] 所拍摄物体为线段    
\par 保持上述公式推导不变,取$k=0$,则可得到
\begin{equation}
\left\{
\begin{aligned}
a&=2lsin\ \theta\\
b&=2l|sin(\theta+\frac{\pi}{3})|\\
c&=2l|sin(\theta+\frac{2\pi}{3})|
\end{aligned}
\right.
\end{equation}
若$\theta\in[0,\frac{\pi}{3})$,此时$b$、$c$满足关系式:
\begin{equation}
\nonumber
\left\{
\begin{aligned}
b+c&=2\sqrt{3}\ lcos\ \theta\\
b-c&=2lsin\ \theta=a
\end{aligned}
\right.
\end{equation}
\par 利用$a$、$b$、$c$表示$l$、$\theta$得:

\begin{equation}
\nonumber
\left\{
\begin{aligned}
l^2&=\frac{(b+c)^2+3a^2}{12}\\
sin^2\theta&=\frac{3(b-c)^2}{3a^2+(b+c)^2}
\end{aligned}
\right.
\end{equation}
\par 回代入$(8)$得$$a=b-c$$
若$\theta\in[\frac{\pi}{3},\frac{2\pi}{3}]$,此时$b$、$c$满足关系式:
\begin{equation}
\nonumber
\left\{
\begin{aligned}
b+c&=2lsin\ \theta=a\\
b-c&=2\sqrt{3}\ lcos\ \theta
\end{aligned}
\right.
\end{equation}
\par 利用$a$、$b$、$c$表示$l$、$\theta$得:

\begin{equation}
\nonumber
\left\{
\begin{aligned}
l^2&=\frac{(b-c)^2+3a^2}{12}\\
sin^2\theta&=\frac{3a^2}{3a^2+(b-c)^2}
\end{aligned}
\right.
\end{equation}
\par 回代入$(8)$得$$a=b+c$$
  \item [2)] 所拍摄物体为圆
\end{itemize}

\end{document}